\documentclass{article}
\usepackage[utf8]{inputenc}
\usepackage{amssymb}
\usepackage{enumerate}
\usepackage{enumitem}
\usepackage{amsthm}
\usepackage{mathtools}
\usepackage{mathrsfs}
\usepackage{amsmath}
\usepackage{hyperref}
\usepackage{eufrak}
\usepackage[czech]{babel}
\usepackage{tikz-cd}
\usepackage{fullpage}
\usepackage{pgfplots}
\theoremstyle{plain}
\newtheorem{thm}{Věta}
\newtheorem{lem}{Lemma}[thm]
\newtheorem{dus}[thm]{Důsledek}
\theoremstyle{definition}
\newtheorem{defn}[thm]{Definice}
\newtheorem{axm}[thm]{Axiom}
\newtheorem{pozn}[thm]{Poznámka}
\newtheorem{znac}[thm]{Značení}
\newtheorem{umluv}[thm]{Úmluva}
\newtheorem{prikl}[thm]{Příklad}
\newtheorem{tvrz}[thm]{Tvrzení}
\newtheorem{zad}[thm]{Zadání}
\usepackage{tikz}
\usetikzlibrary{automata, positioning, arrows}
\tikzset
{
	->, %makes the edges directed
	>=stealth, %makes the arrows head bold
	node distance=3cm, %specifies the minimum distance between two nodes
	every state/.style={thick, fill=gray!10},
	initial text = $ $,
}

\newcommand\cuseq[1]{\mathrel{\overset{\makebox[0pt]{\mbox{\normalfont\tiny\sffamily #1}}}{=}}}
\newcommand\cusar[1]{\mathrel{\overset{\makebox[0pt]{\mbox{\normalfont\tiny\sffamily #1}}}{\longrightarrow}}}
\newcommand\logeq{{\mkern3mu\vDash\mkern-1mu\vline\mkern4mu}}


\renewcommand{\Re}{\mathfrak{R}}
\renewcommand{\Im}{\mathfrak{I}}
\renewcommand\qedsymbol{$\blacksquare$}
\newenvironment{res}[1][\proofname]{%
	\proof[Řešení]}

\title{NI-LOM semestrální úloha}
\author{Michal Dvořák}
\date{\today}

\begin{document}
\maketitle

V této práci se věnujeme problému maximálního toku resp. minimálního řezu v síti. Cílem je porovnat různé formulace a strategie řešení tohoto problému pomocí lineárního programování.

\section{Pojmy}

\begin{defn}
	\textit{Síť} je uspořádaná pětice $(V,E,s,t,c)$ přičemž $(V,E)$ je orientovaný graf (se smyčkama) a $s,t\in V$ dva vyznačené vrcholy - \textit{zdroj} a \textit{stok}. $c\colon E\rightarrow \mathbb{R}^+$ je funkce přiřazující každé hraně $e$ její kapacitu $c(e)$.
\end{defn}
\begin{defn}
	\textit{Tok v síti} $(V,E,s,t,c)$ je funkce $f\colon E\rightarrow \mathbb{R}^+$ splňující:
	\begin{enumerate}
		\item $\forall e \in E\colon 0\leq f(e)\leq c(e)$
		\item $\forall v\in V\setminus\{s,t\}\colon \sum_{(u,v)\in E} f((u,v))=\sum_{(v,w)\in E}f((v,w))$
	\end{enumerate}
	Velikost toku $f$ je
	$$
	w(f) = \sum_{(s,v)\in E}f((s,v))-\sum_{(v,s)\in E}f((v,s))
	$$
	Tok $f$ je \textit{maximální} pokud pro každý tok $f'$ je $w(f)\geq w(f')$.
\end{defn}

Dá se ukázat, že každá síť skutečně má maximální tok i když jsou kapacity libovolná reálná čísla. My se však omezíme na kapacity racionální (resp. celočíselné). Problém nalezení maximálního toku řeší několik standardních algoritmů. Pro srovnání s formulacemi a řešeními pomocí LP použijeme Dinitzův algoritmus. Pro detailní popis Dinitzova algoritmu odkazujeme čtenáře na \cite{labyrint}. Teoretická doba běhu Dinitzova algoritmu je $O(n^2m)$.


Poznamenejme, že problém maximálního toku je úzce spjat s problémem minimálního řezu. Problém nalezení minimálního řezu v síti je nalezení množiny $A\subseteq V$ takové, že $s\in A,t\notin A$ a $\sum_{u\in A,v\notin A}c((u,v))$ je minimální. Na problém minimálního řezu se dá pohlížet jako na duál maximálního toku. I bez teorie lineárního programování se dá ukázat, že velikost minimálního řezu je rovna velikosti maximálního toku v každé síti.


\section{Formulace pomocí lineárního programování}
Problém maximálního toku lze přirozeně formulovat pomocí lineárního programu
\begin{equation}\label{primal_lp}
\begin{array}{ll@{}ll}
\text{max} & \displaystyle\sum_{(s,v)\in E}f((s,v)) - \displaystyle \sum_{(v,s)\in E}f((v,s)) &\\
\text{za podmínek}& \displaystyle\sum_{(u,v)\in E} f((u,v)) - \sum_{(v,w)\in E} f((v,w)) = 0   & & \forall v \in V\setminus \{s,t\}\\
& 0\leq f(e)\leq c(e) & & \forall e\in E
\end{array}
\end{equation}
s proměnnými $f((u,v))$ pro každou hranu $(u,v)\in E$.

Duál programu \ref{primal_lp} je

\begin{equation}\label{dual_lp}
\begin{array}{ll@{}ll}
\text{min} & \displaystyle\sum_{e\in E}y_ec(e)  &\\
\text{za podmínek}& y_e\geq 1   & & e=(s,t)\in E\\
& y_e \geq -1& & e=(t,s)\in E \\
&y_e+y_v\geq 1 & &\forall e=(s,v)\in E,v\notin\{s,t\} \\
&-y_u+y_e \geq -1 & & \forall e=(u,s)\in E,u\notin\{s,t\}\\

&y_v+y_e\geq 0 & & \forall e=(t,v)\in E,v\notin\{s,t\}\\

&-y_u+y_e\geq 0 & & \forall e=(u,t)\in E,u\notin\{s,t\}\\

& y_v-y_u+y_{e} \geq 0 & & \forall e=(u,v)\in E,u\neq s \wedge v \neq t \forall e\in E\\

&y_e\geq 0 & & \forall e\in E
\end{array}
\end{equation}
s proměnnými $y_e$ za každou hranu a $y_v$ za každý vrchol $v\in V\setminus \{s,t\}$.

Tento duál by měl v jistém smyslu odpovídat relaxaci minimálního řezu.  Na první pohled není zřejmé, jak z proměnných $y$ nějaký řez sestavit. Podívejme se na jinou formulaci problému maximálního řezu. Označme $\mathcal{P}$ množinu všech $s$-$t$ cest v síti. 

\begin{equation}\label{exp_lp}
\begin{array}{ll@{}ll}
\text{max} & \displaystyle\sum_{p\in\mathcal{P}}x_p  &\\
\text{za podmínek}& \displaystyle\sum_{p\in\mathcal{P},e\in p} x_p \leq c(e)  & & \forall e \in E\\
& x_p\geq 0 & & \forall p \in \mathcal{P}
\end{array}
\end{equation}

s proměnnými $x_p$ za každou $s$-$t$ cestu. Problémem je, že těch je exponencielně mnoho a tak program není možné v polynomiálním čase ani napsat. Podívejme se ale na duál.

\begin{equation}\label{dual_exp}
\begin{array}{ll@{}ll}
\text{min} & \displaystyle\sum_{e\in E}y_ec_e &\\
\text{za podmínek}& \displaystyle \sum_{e\in p}y_e \geq 1 & & \forall p\in\mathcal{P}\\
&y_e \geq 0& & \forall e \in E
\end{array}
\end{equation}

Ten má proměnnou $y_e$ za každou hranu $e\in E$ ale podmínku za každou cestu $p\in \mathcal{P}$. Myšlenka bude přidávat tyto podmínky do řešiče postupně. Lze nahlédnout, že polynomiálně mnoho podmínek bude stačit pro nalezení optima, protože i Dinitzův algoritmus uvažuje jen polynomiálně mnoho cest z $s$ do $t$ protože oba pracují v polynomiálním čase. Ohodnocení hran $y_e$ můžeme interpretovat jako váhové ohodnocení hran a podmínky za každou cestu jsou splněny všechny pokud vzdálenost\footnote{vzdálenost $d(u,v)$ mezi dvěma vrcholy v orientovaném grafu je délka nejkratší cesty z $u$ do $v$ (případně $+\infty$ pokud žádná neexistuje)} $s$ od $t$ je alespoň $1$. Na hledání nejkratší cesty z $s$ do $t$ lze použít například Dijkstrův algoritmus. Pro rozbor Dijkstrova algoritma opět odkazujeme do \cite{labyrint}.


\subsection{Strategie přidávání cest}
Je otázkou, jakou přesně strategii pro přidávání cest. V prvním kroku, kdy pustíme program \ref{dual_exp} bez podmínek pro všechny cesty, nám řešič vrátí optimum $y=0$ a bude zjevně existovat hodně cest, pro které bude $\sum_{e\in p}y_e<1$. Vyzkoušíme dva přístupy. Pomocí dijkstrova algoritmu najdeme nejkratší cestu z $s$ do $t$. Pokud ta bude délky $\geq 1$, pak máme optimum a vracíme. V opačném přípdě přidáme jednu podmínku pro tuto nejkratší cestu do řešiče a řešič pustíme znovu. V druhém přístupu se pokusíme cest přidat víc. 

TODODODO
Kromě této nejkratší cesty vytvoříme síť rezerv podobně jako v Dinitzově algoritmu a 


\section{Experimenty}
Pro experimentální vyhodnocení byly vygenerovány dvě sady instancí postupně zvětšujících se velikostí. Nejprve řídké grafy s $n$ vrcholy a $dn$ hranami pro $n\in\{50,\ldots,3000\}$ a postupně pro $d\in\{1,\ldots,15\}$.


\bibliographystyle{acm} %don't care about the style really, this one works ...
\bibliography{references}




\end{document}